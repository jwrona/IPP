%%%%%
%Soubor: CHA-doc.tex
%Datum: 19.4.2013
%Autor: Jan Wrona, xwrona00@stud.fit.vutbr.cz
%Projekt: Dokumentace k projektu 2 pro predmet IPP
%%%%%

\documentclass[a4paper, 10pt]{article}[19.04.2013]
  \usepackage[czech]{babel}
  \usepackage[utf8]{inputenc}
  \usepackage[T1]{fontenc}
  \usepackage[text={18cm, 25cm}, left=1.5cm, top=2cm]{geometry}

  \fontfamily{times}

\begin{document}
\noindent
Dokumentace úlohy CHA: C Header Analysis v Python3 do IPP 2012/2013\\
Jméno a příjmení: Jan Wrona\\
Login: xwrona00

\subsection*{Hlavní tělo}
Zde popisuji pohled na program jako celek, popis a vysvětlení jednotlivých tříd, metod
a funkcí následuje.

Při spuštění nejprve dojde na zpracování argumentů příkazové řádky. Následuje kontrola,
zda zadaná vstupní cesta existuje. Pokud ano, následuje větvení programu na část, kdy
je zadán konkrétní soubor a kdy je zadán adresář, případně je použit aktuální pracovní adresář.
Pokud je zadán vstupní soubor, nekontroluje se jeho přípona a volá se funkce \texttt{parseFile},
která zajišťuje potřebný parsing. Při zadaném adresáři je situace mírně odlišná, prochází se totiž
všechny soubory v adresáři i všech jeho podadresářích a ty s příponou \texttt{.h} se dále
zpracovávají stejným způsobem jako tomu bylo u zadaného souboru. Výstup funkce \texttt{parseFile}
se ukládá do seznamu, jenž se předá funkci pro převod do XML. Volitelně se volá funkce pro
odsazení vytvořeného XML. Výsledný textový řetězec se vypíše buď na standardní výstup
(při nezadaném parametru \texttt{---output}) či do souboru.

\subsection*{Popis tříd}
\subsubsection*{{\bf HeaderFile}}
Třída určená pro práci s celým souborem. Z názvu by mohlo plynout, že funkčnost se omezuje
na hlavičkové soubory, ale není tomu tak. Při zadání konkrétního souboru parametrem
\texttt{---input} je soubor zpracován bez ohledu na jeho příponu.

Konstruktor je jediné
místo, kde se pracuje se souborem, ostatní metody díky tomu mají celý obsah souboru 
vždy dostupný v proměnné. Možnosti třídy jsou uzpůsobené potřebám v projektu, ale
implementace i jiných operací nad hlavičkovými soubory je snadno realizovatelná.
Ostatní metody pracují se souborem jako s celkem, odstraňují z něj pro další zpracování
nepotřebné části nebo připravují soubor k dalšímu zpracování.

Popis metod:
\begin{description}
  \item [\texttt{\_\_init\_\_}] Konstruktor. Vytváří se v něm instanční proměnná \texttt{file}, 
  určená pro uložení obsahu souboru. Následně se invokuje dále popsaná metoda \texttt{read()}.
  Kromě proměnné \texttt{self} (v rámci konvence je \texttt{self} vždy první argument metody, 
  proto ji nadále nebudu zmiňovat) má konstruktor ještě argument \texttt{name}, tedy jméno
  vstupního souboru s cestou.
  \item [\texttt{read}] Překopíruje celý obsah souboru do instanční proměnné. V bloku
  \texttt{try} otevře vstupní soubor (název je zadaný v argumentu) v módu pro čtení, zpracuje
  a soubor uzavře. Pokud je během tohoto vyvolána výjimka \texttt{IOError}, došlo k chybě při
  souborové operaci a program je ukončen s příslušným návratovým kódem.
  \item [\texttt{strip}] Odstraní komentáře a deklarace maker. Implementace pomocí regulárních
  výrazů a metody \texttt{re.sub()}, pomocí které nalezené komentáře nahrazuji jednou mezerou
  (aby nedošlo ke spojení např. dvou identifikátorů při výskytu komentáře mezi nimi). Makra, 
  jak jednořádková tak víceřádková, jsou odstraněna.
  \item [\texttt{removeBracesContent}] Odstraní obsah složených závorek. Metoda prochází
  soubor znak po znaku a kopíruje je do nového řetězce. Pokud narazí na levou závorku, 
  inkrementuje počítadlo závorek, naopak pokud narazí na pravou závorku, počítadlo
  dekrementuje. Znaky jsou kopírovány pouze pokud je počítadlo nulové, tudíž se "kurzor"
  nachází vně všech složených závorek. Tímto je dosaženo odstranění jejich obsahu.
  \item [\texttt{findAllFunc}] Najde všechny deklarace a definice funkcí. Realizována
  pomocí regulárního výrazu a metody \texttt{re.findall()}, která najde všechny vzájemně
  se nepřekrývající výskyty. Z důvodu možnosti výskytu znaku konce řádku v deklaraci funkce,
  je nutné nastavit příznak \texttt{re.DOTALL}, který způsobí nestandardní chování znaku
  hvězdička (pasuje i na znak konec řádku). Metoda vrací seznam takto nalezených funkcí,
  přičemž každý prvek obsahuje dvojici: návratový typ + název funkce a seznam argumentů.
  Například funkce \texttt{int f1(double arg1, int*)} by vytvořila dvojici
  \texttt{('int f1', 'double arg1, int*')}.
\end{description}

\subsubsection*{{\bf Function}}
Třída určená pro práci s jedou funkcí.

Popis metod:
\begin{description}
  \item [\texttt{\_\_init\_\_}] Konstruktor. Vytváří potřebné instanční proměnné.
  Při volání volání konstruktoru jsou vyžadovány dva argumenty a to \texttt{func},
  který musí obsahovat dvojici zmiňovanou u metody \texttt{findAllFunc} a druhým
  argumentem, \texttt{name}, by měl být název souboru, ve kterém byla funkce umístěna
  (kvůli výslednému XML souboru).
  \item [\texttt{parse}] Tato metoda pouze invokuje metody pro parsování funkce
  a kontroluje jejich návratové typy.
  \item [\texttt{parseRetVal}] Rozdělí řetězec na návratový typ a název funkce,
  uloží je do příslušných proměnných. Děje se tak opět především pomocí regulárních
  výrazů. Při zadaném parametru \texttt{---no-inline} proběhne kontrola, zda se v 
  deklaraci toto slovo vyskytuje, a pokud ano, tak je funkce ze zpracování vyřazena.
  \item [\texttt{parseArgs}] Rozdělí řetězec na jednotlivé argumenty - konkrétně
  jejich datové typy a jména. Dále se zde kontroluje výskyt výpustky, pokud je nalezena
  nastavuje se příznak, který se projeví ve výsledném XML. Metoda neumí správně zpracovat
  argumenty zadané bez jména.
\end{description}

\subsection*{Popis funkcí}
\subsubsection*{{\bf argumentParser}}
Funkce obstarává zpracování argumentů předaných programu na příkazové řádce.
Využívá možnosti modulu \texttt{argparse}. I přesto, že jsou možnosti tohoto
modulu velice rozsáhlé, ne vždy plně vyhovoval. Například kvůli kontroly, zda se
každý argument vyskytuje maximálně jednou byla vytvořena třída \texttt{MyAction},
která je předávána jako parametr metodě \texttt{add\_argument()}, a tuto kontrolu
zajišťuje. Už zpracované parametry jsou uloženy do proměnné jako datový
typ slovník.

\subsubsection*{{\bf parseFile}}
Nalezne všechny deklarace a definice funkcí v souboru, vrátí seznam instancí
třídy \texttt{Function}. Funkce využívá především dříve popsaných tříd
\texttt{HeaderFile} a \texttt{Function}, které instanciuje a zasílá jim
potřebné zprávy.

\subsubsection*{{\bf generateXml}}
Funkce vygeneruje textový řetězec obsahující výsledné XML. To je neformátované,
ve formě čitelné především pro strojové zpracování. Je zde využito modulu
\texttt{xml.etree.ElementTree}.

\subsubsection*{{\bf prettifyXml}}
Zpracuje vstupní string obsahující validní XML do čitelnější podoby.
Výsledný string obsahuje odsazení o zadaný počet mezer. Využívá se modulu
\texttt{xml.dom.minidom}. 
\end{document}
