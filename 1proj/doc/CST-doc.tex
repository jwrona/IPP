%%%%%
%Soubor: cst.tex
%Datum: 12.3.2013
%Autor: Jan Wrona, xwrona00@stud.fit.vutbr.cz
%Projekt: Dokumentace k projektu 1 pro predmet IPP
%%%%%

\documentclass[a4paper, 10pt]{article}[12.03.2013]
  \usepackage[czech]{babel}
  \usepackage[utf8]{inputenc}
  \usepackage[T1]{fontenc}
  \usepackage[text={18cm, 25cm}, left=1.5cm, top=2cm]{geometry}

  \fontfamily{times}

\begin{document}
\noindent
Dokumentace úlohy CST: C Stats v Perl do IPP 2012/2013\\
Jméno a příjmení: Jan Wrona\\
Login: xwrona00

\subsection*{Hlavní tělo}
Hlavní část programu jsem umístil vně všechny subrutiny, to je možné díky tomu,
že v perlu není povinná žádná hlavní funkce (subrutina) a kód se začne vykonávat
(s výjimkami) od prvního řádku. Do této části jsem v prvé řadě umístil funkce
\texttt{use} zajišťující import funkčnosti jiných modulů. Pro tento program jsem
použil modul \texttt{Cwd} pro zjištění aktuálního pracovního adresáře a absolutní
cesty určitého souboru, dále potom modul \texttt{common::sense}.
Také se zde nacházejí deklarace proměnných, které jsou potřeba v rámci celého programu.
Především to je pole klíčových slov (detailnější popis později), text nápovědy, výčet
návratových kódů, file descriptory, pole výsledných hodnot a jiné. Z důvodu přehlednosti
jsem funkční kód v této sekci maximálně omezil na základní větvení, zda je na vstupu
soubor či adresář a sadu volání základních subrutin (zpracování parametrů, otevření
souboru, požadovaná akce nad soubory, \dots) a jediné ukončení programu s návratovým kódem 0, 
tedy úspěšné ukončení.

\subsection*{Popis subrutin}
Program budu popisovat v přibližně takovém pořadí, v jakém se vykonává. Začnu tedy
subrutinou \texttt{makeParam()} - ta staví především na modulu \texttt{Getopt::Long}.
Tento modul jsem náležitě nakonfiguroval, jeho defaultní nastavení požadavkům ze
zadání nevyhovovalo. Použita je klíčová funkce modulu, \texttt{GetOptions()}, která
na základě parametrů na příkazové řádce nastaví hodnoty svým parametrům. Nicméně
i přes takto zjednodušenou práci jsem musel řešit řadu problémů, jako jsou zakázané
kombinace některých přepínačů, vícenásobný výskyt parametrů, nutnost zadání
povinných parametrů a podobně. Pokud jsou tyto kontroly úspěšné, je program schopný
určit, jaká akce se bude dále vykonávat (co bude program počítat).

Podle argumentu přepínače \texttt{---input} se rozpozná, zda se bude zpracovávat
jeden vstupní soubor, či adresář. Pokud je tento přepínač nezadán, bude se pracovat
s aktuálním adresářem. Samozřejmostí jsou testy na správné otevření souboru, doprovázené
ukončením programu při jejich neúspěchu.

Při zadaném konkrétním souboru je volána subrutina \texttt{doAction()}, která obsahuje
jednoduchý switch (modul \texttt{Switch}), jehož funkce je zavolat subrutinu dle
vstupního přepínače. Pokud je zadán adresář, nejprve dojde na volání subrutiny
\texttt{goThroughDirs()}, která, jak název napovídá, projde zadanou adresářovou
strukturu. Seznam podadresářů získá vestavěnou funkcí \texttt{glob()}, zanořování je řešeno
rekurzí. Funkcí \texttt{glob()} je následně získán seznam souborů s příponami
".c .h" a nad každým je zavolána dříve zmiňovaná subrutina \texttt{doAction()}.

Konečně se dostávám k jádru programu, které tvoří subrutiny
\texttt{count\{Keywords, Operators, Identifiers, Comments, Pattern\}()}. Ty všechny
mají určité společné rysy popsáné následovně. Vstupní soubor je zpracován řádkově,
je tedy snížená spotřeba paměti při velkých souborech. Dále je to způsob zpracování
textu - odstraňováním nechtěných částí vstupního kódu (mohou to být makra, komentáře, 
textové řetězce nebo znakové literály). Tento způsob se mi jevil jako nejjednodušší
a tudíž nejefektivnější. Většina těchto operací je řešena za pomocí
regulárních výrazů, ať už se substitucí či nikoliv. Všechny tyto subrutiny vracejí
výsledný počet pro jeden soubor.

Klíčová slova počítá subrutina \texttt{countKeywords()} a to velice jednoduchým
způsobem. Nejprve jsou odstraněny nežádoucí části kódu - v tomto případě to jsou
komentáře, makra, řetězce, znakové literály a prázdné řádky (pouze kvůli efektivitě).
Klíčová slova jsem umístil do pole a pro každé z těchto slov postupně pro každý řádek
použitím regulárního výrazu \texttt{\textbackslash b\$word\textbackslash b} tato
slova počítám. Pro operátory existuje subrutina \texttt{countOperators()}, jejíž
funkčnost je podobná - "odstraňuje" z kódu stejné nežádoucí části, jen k nim přidává
tečky z desetinných čísel a tři tečky z deklarací funkcí, protože to nejsou operátory
(ale používají stejný symbol jako přístup do struktury, což operátor je). Pole
operátorů bylo nutné seřadit počínaje těmi s nejvíce znaky. Jinak by například
operátor bitového posuvu mohl být počítán jako dva operátory \texttt{>}. Při takto
seřazených operátorech dostávají prioritu ty "delší". Pro použití v regulárním
výrazu bylo potřeba některé znaky opatřit zpětným lomítkem - řeší to vestavěná funkce
\texttt{quotemeta}. Hledání a počítání identifikátorů (jak jinak než subrutinou
\texttt{countIdentifiers()}) je opět řešeno odstraněním komentářů, maker, řetězců
a literálů, navíc se k nim přidávají klíčová slova. Je tomu tak proto, že detekce
identifikátorů by je započítala, což by byla chyba. Ze specifikace C99 je identifikátor
posloupnost znaků začínajících velkým či malým písmenem nebo podtržítkem, následovaná
libovolným počtem množiny těch znaků doplněné o dekadické číslice. Použil jsem
následující regulární výraz:
\texttt{\textbackslash b[\textunderscore a-zA-Z][\textunderscore a-zA-Z0-9]*\textbackslash b}.
Znaky v komentářích počítá subrutina \texttt{countCommnets()}, ať už jde o víceřádkové
či jednořádkové komentáře. Ze zadání plyne, že konec řádku se počítá jako jeden znak, 
i když je značen \texttt{CRLF}. Pro jednoduchost veškeré \texttt{CRLF} nahrazuji
\texttt{LF}. Znaky jsou počítány vestavěnou funkcí \texttt{length()}, které jsou
argumentem substringy vytvořené regulárními výrazy. Důležitá je detekce počátku
víceřádkového komentáře, od něhož se všechny znaky započítávají, dokud není detekován
konec komentáře. Poslední z těchto subrutin, \texttt{countPattern()}, počítá počet
výskytů zadaného textového řetězce. Pro případ, že by obsahoval znaky kolidující s
kontrolními znaky regulárního výrazu jsem opětovně využil funkce \texttt{quotemeta()}.

Předchozí subrutiny využivají pomocné rutiny \texttt{remove\{Comments, Macros, StrChar\}()},
jejichž funkce je z názvu jasná. Všechny jsou implementovány s využitím regulárních výrazů a 
očekávají jediný parametr a to text (řádek). Funkce odstraňující komentáře pracuje jak
s jednořádkovými tak i s víceřádkovými komentářemi, \texttt{removeMacros()} zpracovává
makra preprocesoru počínající operátorem \texttt{\#}, pokud je poslední znak makra zpětné
lomítko, je za makro považován i následující řádek. Poslední z těchto subrutin, 
\texttt{removeStrChar()}, ostraňuje textové řetězce, tedy veškerý obsah ohraničený
znaky \texttt{"} včetně těchto znaků a znakove literály ohraničené znaky \texttt{'}
také včetně.
\end{document}
